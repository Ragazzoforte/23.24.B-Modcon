\begin{center}
    Is het gebruik maken van een AI chatbot goed of fout?
\end{center}

\vspace{1cm}


Als je kijkt naar de principes van de deugdethiek, kun je daar vraagtekens bij zetten. Tegenwoordig gebruiken veel mensen een AI-chatbot, maar is dat wel verstandig?

\begin{itemize}
    \item Is het gebruik maken van een ai chatbot wel eerlijk?
    \item Is het gebruik maken van een ai chatbot wel rechtvaardig?
    \item Is het gebruik maken van een ai chatbot wel wijs?
\end{itemize}

Het zijn allemaal zaken waarover we misschien even moeten nadenken. Is het wel verstandig om een AI-chatbot te gebruiken bij het schrijven van een verslag? Als je het aan mij zou vragen, zou ik nee zeggen. Echter, als je me zou vragen of ik een AI-chatbot gebruik bij het maken van een verslag, zou ik ja zeggen. Ook al weet ik dat het misschien niet verstandig is om gebruik te maken van een AI-chatbot, doe ik het toch.

\vspace{.5cm}
Waarom? 

Gemakzucht! 
\vspace{.5cm}

Eerlijk is eerlijk, gemakzucht dient vaak de mens, maar als we eens goed kijken naar wat zich nu allemaal schuilt achter zo'n AI-chatbot, is dat niet echt iets om trots op te zijn. Vaak zijn het grote techbedrijven die zichzelf de hemel in prijzen door te zeggen dat het allemaal open source is en dat iedereen onder de motorkap kan kijken. Maar als je het hendeltje probeert te vinden om de motorkap te openen, is die nergens te vinden. Niemand weet wat de beperkingen en risico's zijn. Spreekt zo'n AI-chatbot wel de waarheid? Probeert een AI-chatbot je te misleiden?

\vspace{.25cm}

Het gebruik van een AI-chatbot voor bijvoorbeeld het maken van een systeem, het regelen van een systeem of het schrijven van een verslag is niet per se goed of fout. Het gaat erom dat je er bewust over nadenkt. Neem jij zomaar over wat een AI-chatbot zegt of denk je er ook nog echt over na? Eén ding weet ik wel: als jij kiest voor het eerste (de makkelijke optie), dan maak je jezelf alleen maar dommer en zeker niet slimmer. Ook moet je jezelf afvragen: is dat nog wel eerlijk? Iedereen leert door vallen en opstaan. Je ontdekt iets, je leert iets en er mislukt iets, en daar word je beter van. Als jij al deze stappen door een AI-chatbot laat doen, leert alleen de chatbot en leren wij als mensen niet. Op deze manier zal de AI-wereld ons op een gegeven moment flink inhalen. Iets wat ik liever niet zie gebeuren, is dat wij als mensen niet meer de touwtjes in handen hebben. Om dat te voorkomen denk ik dat we wel met beide voeten op de grond moeten blijven staan en niet altijd de makkelijkste weg moeten kiezen door alles te laten invullen door AI-chatbots.

\vspace{.25cm}

Want wat is nou eigenlijk nog echt?

\cite{cm}
\cite{De_Volkskrant}
\cite{Filosofie}
